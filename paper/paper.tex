\documentclass[12pt,a4paper]{cibb}
\setlength{\parindent}{0pt}

% Fallback definition for missing \@ordinalM
\makeatletter
\providecommand{\@ordinalM}[2]{#1}
\makeatother

\usepackage{etoolbox}
\makeatletter
\patchcmd{\@startsection}{\@afterindenttrue}{\@afterindentfalse}{}{}
\makeatother

\usepackage{subfigure,graphicx}
\usepackage{amsmath,amsfonts,latexsym,amssymb,euscript,xr}
\usepackage{booktabs}
\usepackage[nodayofweek]{datetime}
\usepackage{hyperref}
\usepackage{fmtcount}
\usepackage[english]{datenumber}
\usepackage[absolute]{textpos}



\usepackage[table]{xcolor}
\usepackage{color,colortbl,tabularx}

\usepackage[english]{babel}
\usepackage[protrusion=true,expansion=true]{microtype}
\usepackage{amsmath,amsfonts,amsthm}
% \usepackage[pdftex]{graphicx}
\usepackage{pifont}

\def\red{\color{red}}
\def\black{\color{black}}
\def\blue{\color{blue}}
\def\magenta{\color{magenta}}



\definecolor{LightBlue}{rgb}{0.88,0.9,0.9}

\newcommand\blfootnote[1]{%
  \begingroup
  \renewcommand\thefootnote{}\footnote{#1}%
  \addtocounter{footnote}{-1}%
  \endgroup
}

\title{\Large $\ $\\ \bf GoE (Graph-of-Experts): Trainable Graphs for Hierarchical Multi-Scale Reasoning}

\author{\large Michael Vaden$^{1, 2}$}
\address{\footnotesize $\ $\\$^1$ Independent Researcher, Atlanta, USA
$\ $\\$^2$ College of Computing, Georgia Institute of Technology, Atlanta, USA
\newline
$^*$corresponding author: mvaden6@gatech.edu, michaelvaden.mjv@gmail.com
}


\abstract{\small Mixture of Experts, Graph Networks, Transformers \normalsize
\\[17pt]
{\bf Abstract.} 

-propose a Graph of Experts (GoE) architecture 

-general framework for adaptive, compositional problem-solving 

-each expert represents a specialized subnetwork 

-edges between experts are learnable pathways encoding how info should flow 

-gating policy learns which experts to invoke and how to traverse the expert graph 

-gating enables dynamic routing, feedback, and recombination of partial results 

-single pass -> structured reasoning 

-blueprint for modern hierarchical intelligence 

-efficient computation along graphs 

-could coordinate reasoning across multi-modal domains 

-need experiment results 
\\
}

\begin{document}


\thispagestyle{myheadings}
\pagestyle{myheadings}
\markright{\tt}%check year

\textbf{1 Introduction}

-hierarchical problems require systems that adapt across levels of abstraction 

-vesuvius scrolls interesting because of diversity of challenges

-conventional deep networks such as UNets or transformers apply the same operations to every region, lacking mechanisms to dynamically compose specialized behaviors 

-reframe modular neural routing as a learnable graph 

-each node goes with an expert trained for distinct subtask 

-in the vesuvius case, geometry reconstruction, fiber orientation, ink segmentation 

-directed nodes encode transition policies governing how information propagates 

-GoE learns structured pathways through the expert graph, discovering intermediate representations to solve complex, multi-scale problems 

-unifies specialization and coordination 

-gating mechanism dynamically selects experts and edge transitions 

-allows for self-organization into meaningful workflows  

-Add experiment info

\hfill

\textbf{2 Background}

-Recent progress in adaptive architectures explores dividing neural computation into specialized components  

-MoE paradigm trains multiple experts in a specific data regime, while a gating network selects which to activate per input 

-Improves efficiency but remains flat due to one pass 

-Google’s mixture of recursions introduces recursive expert calls, anbling dynamic reasoning chains, these are typically linear or sequential, optimized for symbolic or temporal reasoning tasks 

-They lack explicit modeling of relationships between experts themselves, for instance, how information should transition between specialists handling distinct subproblems 

-Each edge encodes transition policies 

-Gating mechanism operates not as a simple router but as a graph traversal policy 

-Trained to discover efficient and semantically meaningful pathways across experts 

-Learns compositional workflows rather than isolated specializations 

-Mention vesuvius relation 

\hfill

\textbf{3 Model Architecture}

Here we formalize GoE as a modular pipeline. We cleanly separate input stems, modality-sepecific adapters that transform
raw data into tokens with positions/scale, from a task-agnostic encoder that adds local/global context and emits per-token
content embeddings $h$ (for experts) and routing features $e$ (for the graph router). The GoE core then performs sparse
traversal over a library of lightweight experts under a learned graph router, allocating computation by difficulty and
recording path provenance. Task heads are pluggable and minimal (used only for supervision/inference), so stems and heads
change with the domain while the GoE core is identical. 

\hfill

\textbf{3.1 Input Stem}

Given a raw input $x$, the input stem should produce some sequence of tokens, $T \in \mathbb{R}^{B \times N \times d}$, as
well as the auxiliary routing features for the graph router, $A \in \mathbb{R}^{B \times N \times d_a}$. The input stem is
the only component of the system that is modality-specific, as it is the modality adapter for the problem.

\hfill

\textbf{3.2 Encoder}

The encoder provides global context for the auxiliary routing data with modality-specific methods. Given input from the
input stem, the encoder produces multi-level feature maps, $F^0, \ldots F^L$, with $F^L$ being the least dimensional
feature map. This is executed by using repeatable G-Blocks with residuals. The encoder then installs a slight attention
mechanism to ensure multi-modality multi-reasoning. It should be fused minimally via a content embedding. Routing features
should also be extracted from the auxiliary routing features.

\hfill

\textbf{3.3 Graph Router}

Experts $\{f_k\}_{k=1..M}$ are generally trainable neural circuits with parameter efficient adapters for modality hints.

\hfill

\textbf{3.4 Experts}
-include training section

\hfill

\textbf{3.5 Traversal}

\hfill

\textbf{3.6 Task Heads and Losses}

\section*{Conflict of interests}
\label{sec:CONFLICT-OF-INTERESTS}
The authors should declare here any potential conflicts of interests.

\section*{Acknowledgments (optional)}
\label{sec:ACKNOWLEDGMENTS}
The authors would like to thank XXX and YYY for their helpful feedback.

\section*{Funding (optional)}
\label{sec:FUNDING}
This work was supported by the XXX agency (grant number: XXX).

\section*{Availability of data and software code (optional and strongly suggested)}
\label{sec:AVAILABILITY}
Our software code is available at the following URL: XXX.
\newline
\indent Our dataset is available at the following URL: XXX.


\footnotesize
\bibliographystyle{unsrt}
\bibliography{bibliography_CIBB_file.bib} 
\normalsize

\end{document}
